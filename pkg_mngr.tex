% Package manager how-to:
\documentclass[12pt, letterpaper]{article}

\usepackage[top=1in, bottom=1.5in, left=.5in, right=.5in]{geometry}
\usepackage[colorlinks=true,urlcolor=blue]{hyperref}
\usepackage{enumerate}

\begin{document}

\begin{center}
  {\LARGE \textbf{Package Managers in Linux-Like Systems}}\\
  {\large Your friend in installing software}\\
\end{center}

\section{Background}
The open-source software environment for Linux-like operating systems provides
an expansive library that can address almost any issue you may have as a
scientitific programmer.  However, unlike commercial software for Windows or
OS X systems, there's probably not a point-and-click installer to get the
software to work on your computer.  The manual installation process begins with
obtaining the source code, compiling it so that it works with your machine,
then placing it into the right directory so that it can be found by your
system when it is called.  Compounding these complications is the issue of
dependencies- software frequently relies on other pieces of software to run.
Therefore, each installation requires you identifying the dependency tree-
the heirarchy of dependencies you must satisfy in order to successfully
compile the particular item you want.  This can be a very difficult problem
for even the most experienced users.

Fortunately, there is a powerful solution- package managers.  Package managers
are programs that access code repositories, recognize dependency trees, and
automatically fetch and compile code in a very straight forward manner.  If
you are using a Linux system, you have a package manager already set up and
ready to go.  If you are on a Mac, then you just need to install one of
several available for OS X systems.  If you're in Windows, then you should be
reading a different document.

\end{document}
