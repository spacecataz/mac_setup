% Package manager how-to:
\documentclass[12pt, letterpaper]{article}

\usepackage[top=1in, bottom=1.5in, left=.5in, right=.5in]{geometry}
\usepackage[colorlinks=true,urlcolor=blue]{hyperref}
\usepackage{enumerate}

% Add watermark to document:
\usepackage{draftwatermark}
\SetWatermarkText{DRAFT}
\SetWatermarkScale{7}

\begin{document}

\begin{center}
  {\LARGE \textbf{Your Mac as a Linux Box}}\\
  {\large A guide to open-source software on OS X}\\
  {\large \today}\\
\end{center}

\section{Background}
In the late 90s, MacOS - the operating system used on Macintosh computers - had
become a bloated, clumsy mess.  Apple's solution was to turn to Unix-based
technology developed by the \href{https://en.wikipedia.org/wiki/NeXT}{NeXT}
company (founded by Steve Jobs during his ``sabatical'' from Apple).
The result was the now widely used OS X, the most widely used operating system
outside of Microsoft's Windows platforms.
Because OS X is, at its core, a Unix system, it was quickly adopted by
(amongst others) scientists who need both the open-source and command line
capabilities of Unix/Linux systems, but also need access to popular
point-and-click programs such as MS Office, MatLab, and others.  OS X is
therefore the ``best-of-both-worlds'' for scientists.

The move from a traditional Linux systems to OS X comes at a price, however.
Apple insists on several non-standard implementations of languages (e.g.,
python), window managers, and file system layouts.  This means that to
unlock the full Unix/Linux-like capability of your Mac, you need to do a bit
more setup work.  This guide attempts to get you set up so that you can start
coding like a Linux pro!

Note that this process will take \emph{time}.  Getting software, installing
packages, and getting up-and-running requires plenty of time to download and
compile libraries and packages.  Make sure you're ready before you start.

\section{Accessing the Terminal}
Open up a Finder window and navigate to
{\tt Applications/Utilities/Terminal.app}.  Start that up.  Tadaa!  That's all!

...I lied, it's never that easy.  You can happily use the terminal and do
most of the things you need to do, but I \emph{strongly} recommend obtaining
an \href{https://en.wikipedia.org/wiki/X_Window_System}{X Windows System}
(or just X11 for short) compatable terminal.  That is available via
\href{http://www.xquartz.org/}{XQuartz}, a project dedicated to creating a
pure X11 experience on your Mac.  Download and install the XQuartz terminal.
Trust me.  It's better.  Once you have it installed, open a terminal and
start playing.

A critical thing to understand in X11 is the copy-paste mechanics. In Linux,
anything you highlight is copied to the clip board and can be pasted
via a middle-click of the mouse.  In OS X and XQuartz, \textbf{middle click
  of the mouse will paste any highlighted text in an X11 app OR text copied
  to the standard OS X clipboard, which ever happened last.}  If you do not
have a middle button on your mouse, an equivalent input is to hold the option
key and left-click.  If this behavior
is not working, be sure to check ``Emulate three button mouse'' in
XQuartz's preferences.  

This mechanism seems clunky at first, but is much
faster than the usual command-C command-V keystrokes.  Think about it:
you usually need to put your hand on the mouse, highlight text, then move
your hands to the keyboard to press command-C to copy it.  Then, back to the
mouse to select where you want the text to be placed, then back to the
keyboard for command-V paste.  With the middle-click paste, your hands stay on
the mouse the whole time.

An important trick is that you can customize what your terminal window looks
like.  Under the ``Applications'' menu in XQuartz, click on ``Customize...''.
You will see the default ``Terminal'' command, which simply calls xterm.
If you edit the ``command'' column of an existing row, or create a new row and
edit the command column of that, you can pass options to the xterm command
that changes the font, color, and other behavior to your terminal.  For example,
\begin{verbatim}
xterm -fa Monaco -fs 16 -geometry 60x13
\end{verbatim}
creates a window that is 60 by 13 characters in size while changing the font and
font size for better clarity.  If you click ``Add Item'', you can create a new
command for more customization.  Suppose you want a window that automatically
logs into a certain machine and sets new colors so you know that window is
different than your default.  Do this:
\begin{verbatim}
xterm -fg Gold -bg DarkBlue -T Umich -e ssh -YC yourname@login.itd.umich.edu
\end{verbatim}
Now, when you select this option from the ``Applications'' menu, you'll get
a window that automatically logs into your Umich webspace.  For more options,
see the man page for the
\href{http://linux.die.net/man/1/xterm}{xterm command}.  

\section{Text Editors}
At this point it is critical that you have a go-to text editor within your
X11 environment.  The two most popular are Emacs and Vi.  They are also both
already installed on all Linux-like machines, including your Mac.
I'm not going to go in to which you should use or shouldn't use, or the
advantages of each.  However, I prefer Emacs and find it to be very powerful.
\href{https://www.gnu.org/software/emacs/refcards/pdf/refcard.pdf}{An
  Emacs command cheat-sheet can be found here.}  At this point, pick either
Emacs or Vi and do a quick internet search on how to start, edit text, save
a file, and exit.  Be sure you can do this before moving forward.

Let's assume that you will be using Emacs.  Start an Emacs session by
typing ``Emacs'' at the command line prompt.  The terminal then enters Emacs
mode, and you can begin typing in a blank text file.  If you want to save
what you're typing, press C-x C-s (that's control and x at the same time,
then control and s at the same time.)  Emacs will ask you to name the file.
To quit Emacs, press C-x C-c.  At this moment, the version of Emacs you are
using has no point-and-click interfaces (called Graphical User Interfaces, or
\emph{GUIs}).  In Section \ref{sec:ports}, we'll see how to rectify this.

\section{Editing the Configuration File}
An important and unavoidable part of customizing \emph{any} Linux-like
environment is editing the configuration files.  You'll find these in your
home directory (type {\tt cd} to get there) by typing {\tt ls -a}.  Most of
those files that start with a dot are config files.  Look for {\tt .profile},
{\tt .bashrc}, or {\tt .cshrc}.  The first two are for users of BASH, the
Bourne-Again Shell.  The latter is for C-Shell users.  What one you edit
depends on what shell you are currently using.  To figure that out, type
{\tt echo \$0}.  \textbf{This guide assumes you are using BASH, as it is the
  default for modern OS X and Linux machines.}

The shell configuration files are a series of commands that are run when you
start a new terminal.  You can use these to set your search path, set aliases,
configure code repositories, and more.  You'll be constantly updating your
shell config files as you expand your software library and customizing your
computing environment.

Let's do a few things to customize the shell to your preferences.  Let's start
by finding what configuration files you have.  Use {\tt ls -a} in your home
directory to see if any of the following files exist:
\begin{itemize}
\item {\tt .profile}
\item {\tt .bashrc}
\item {\tt .bash\_profile}
\end{itemize}

The difference between these three is subtle and sometimes system dependent.
Different configuration files are used by Bash in different situations, so
you may find that one works and others appear to have no effect.
It appears that, for OS X, the file that seems to be the most robust is
{\tt .profile}.

Open one of the above files using emacs (e.g., {\tt emacs .profile}.  If
none of the above exist, using {\tt emacs .profile} will create that file.
Move your cursor towards the end of the file, and add this text:
\begin{verbatim}
# Set default text editor:
export EDITOR='emacs -nw'
\end{verbatim}
Note the comment, preceeded by a hastag, so that you can remember what this
line does in the future.
If you're a Vi user, replace {\tt 'emacs -nw'} with {\tt 'vi'}.  This command
changes the default text editor that will open for programs such as
Visudo.

\subsection{Search Paths}
\label{sec:path}
The next thing you'll want to do is change your \emph{search path}.  This is
the set of directories that your computer searches everytime you enter a
command.  When you type {\tt ls} to look at the contents of the current
directory, BASH is quietly searching for an executable program called ``ls''.
If you want to know where BASH finds this command, use {\tt which} at the
shell prompt:
\begin{verbatim}
which ls
\end{verbatim}
This should return {\tt /bin/ls}.  To see your search path, type the following
at the shell command prompt:
\begin{verbatim}
echo $PATH
\end{verbatim}
You should now see a colon-separated list of directory paths.  Let's edit
your config file to change your search path.  Suppose you have a directory
full of scripts that you would like to access from any location.  Make this
directory now by typing {\tt mkdir myscripts} from your home directory.
Now, add these lines to your config file:
\begin{verbatim}
# Add scripts directory to search path
export PATH=$PATH:~/myscripts
\end{verbatim}
..where \textasciitilde is shorthand for your home directory path.  This line
keeps the default search path in tact and appends a new entry to it.  The
shell will always search the path list in order, so {\tt ~/myscripts} will be
searched last.  If there are multiple files with the same name in different
searched files, the first one found will be used.  Therefore, if you want
your script directory to be searched first, simply change the order like so:
\begin{verbatim}
# Add scripts directory to beginning of search path
export PATH=~/myscripts:$PATH
\end{verbatim}

\subsection{Activating \& Checking Config Files}
Now, let's get BASH to recognize the changes you've made.
To make these changes take effect immediately, type
{\tt source \textasciitilde/.profile} or {\tt source \textasciitilde/.bashrc}.
Otherwise, they will take effect once you open the next terminal window.

Test to make sure that these changes by opening a new terminal and using these
commands at the prompt:
\begin{verbatim}
echo $SHELL
echo $EDITOR
\end{verbatim}
The first line should return your colon-separated list of directories; make sure
that your new directory {\tt \textasciitilde/myscripts} is in that list.
The second line should print out your default text editor.  Make sure it lists
the one you set.

If these changes do not seem to be taking effect, there are several things that
could be wrong.  If you're seeing an error message when you open a new window,
that means that you've made a mistake in your config file.  One-by-one,
comment out the lines you added and open a new terminal between each change to
find out exactly what line is giving you troubles.  Try entering that line
into the command line prompt directly and change it to figure out how to get it
to work.  If there is no error but the config file changes do not seem to have
any effect, it is likely that your shell is not using the configuration file
you are chosing to edit.  Use the {\tt echo} command in each of your files
to cause them to print messages to the screen:
\begin{verbatim}
echo 'This is your bash_profile file checking in!'
\end{verbatim}
Of course, customize the message to match each file's name.  Now, open a new
terminal window to see what config file is being used and move your commands
to that one.  Finally, if none of the above items work, turn to Google.  Be
sure you know what shell you are using ({\tt echo \$0}) as you search.

\section{Sudo Access}
Installing software requires administrator access.  You know this already:
when you install new software on your Mac, it asks for your admin user and
password.  At the command line, you do this via {\tt sudo}, which stands
for \emph{super user do}.  Any command preceeded by {\tt sudo} at the
terminal prompt will be executed with administrator rights.  This is required
for installing new software from the command line prompt.

Only some users can use {\tt sudo}, and you will need it moving forward.  To
get sudoer access, you must change the \emph{sudoer's file}.  The easy way to
do this is first log into OS X using an administrator's account.  Then,
from the command line, use the {\tt visudo} command:
\begin{verbatim}
export EDITOR='emacs -nw'
sudo visudo
\end{verbatim}
Enter your administrator password when asekd.  This will open the sudoer's
file for editing in your preferred editor (set by the {\tt EDITOR} environment
variable.  Note that because we are now using a different account (i.e., your
admin account), we have to set this variable again!
Find the lines inside of the sudoer's file that look like this:
\begin{verbatim}
root    ALL=(ALL) ALL
\end{verbatim}
Copy that line, but replace 'root' with your regular user name.
\textbf{Be sure to leave the root line intact or you will have big problems!}
The result should look something like this:
\begin{verbatim}
root    ALL=(ALL) ALL
dwelling    ALL=(ALL) ALL
\end{verbatim}

Save and exit your text editor and return to your normal computer account.
Test your sudo access:
\begin{verbatim}
sudo echo "it works!"
\end{verbatim}
Sudo will ask for your password and then, if everything was done correctly,
print out your message to screen.

\section{Installing MacPorts}
\label{sec:ports}
Open-source software is the backbone of any linux-based system.  Getting new
software isn't done like other platforms, where you download installers or
use an app store.  Rather, you depend on a \emph{package manager}.  A
package manager allows you to search available open-source software,
automatically download a program you want to install and all of its
dependencies, and install everything you need.

There are options for OS X.  The three big ones are, in order of age,
\href{http://www.finkproject.org/}{Fink},
\href{https://www.macports.org/}{MacPorts}, and
\href{http://brew.sh/}{Homebrew}.
I recommend MacPorts because it is well established and the packages are
well maintained.  The following guide is a crash-course to installing and
using MacPorts.  For a more complete description, check out the
\href{https://guide.macports.org}{official guide} and don't forget to
\href{http://www.google.com}{GDS} when you get stuck.

\subsection{Installing and Configuring MacPorts}
Start by downloading and installing MacPorts from the
project website.  Follow the instructions on the website, including installing
XCode from Apple.  Part of the installation process is editing your search
path to include {\tt /opt/local/bin}, where MacPorts keeps software.
Be sure to check that your shell config file was correctly edited by the
installer.  A recommended change is to put the MacPorts directory at the
beginning of your search path, like so:
\begin{verbatim}
export PATH=/opt/local/bin/:$PATH
\end{verbatim}
This prevents your shell from finding Mac's default programs before your
shiny new MacPorts versions.

\subsection{Installing Code}
Let's use MacPorts.  Note that there are more in-depth guides to doing this,
such as the one provided on the MacPorts website.
Start by finding a package you want to install, such as
emacs.  To list ALL available software, use this command: {\tt port list}.
That's a long list!  let's search for just emacs:
\begin{verbatim}
port search emacs
\end{verbatim}
This returns a list of everything that has ``emacs'' in the name or description.
While this list is much shorter than just listing everything (50 results at the time
of writing), it's still too much.
I've found that {\tt grep} can help create more concentrated searches:
\begin{verbatim}
port list | grep -i emacs
\end{verbatim}
This will print out all versions of emacs that are available.  The plain-old
{\tt emacs} seems good enough.  For each sotware package, there are also
\emph{variants}, which allow customizations of your install.  We can
see our variants for emacs like so:
\begin{verbatim}
port variant emacs
\end{verbatim}
The {\tt x11} variant gives us emacs in its own window, which is what I
prefer!  So finally, let's install it:
\begin{verbatim}
sudo port install emacs +x11
\end{verbatim}
...and MacPorts goes to work. This will take a while, but it does it
eventually.  If anything ever goes wrong, find the exact error message and
search it online.  The MacPorts user community is very large and active.
Very rarely is there a problem that you have that someone else hasn't already
solved.

\subsection{Activating \& Selecting Software}
Inevitably, you'll have more than one version of a program installed via
MacPorts.
There are many reasons for this, but most commonly this is due to different
dependencies for different software.  For example, one piece of software
may require Python 2.7 to install, so MacPorts installs Python 2.7 as a
dependency
while installing your target software.  A different program may require
Python 3.6
as a dependency, so MacPorts installs that as well.  Worse yet, some libraries
require prequisites with one set of variants; other libraries require the same
prequisites with \emph{different} variants!  When you call that software from
the command line, which one are you using?

There are two tools to help you sort things out.  The first is \emph{activate},
which allows you to choose between two identical pieces of software that have
different variants.  As this is not a common procedure, I'll simply
\href{https://guide.macports.org/#using.port.upgrade}{link to a descriptive
  example here}.  The other tool is \emph{select}, which is frequently used
to help decide which version is the ``default'' version when called from
the command line.

An illustrative example comes in the form of the Gnu C Compiler, {\tt gcc}.
This is used frequently, both by the user and by many other pieces of software.
I can see what versions are installed via the following syntax:
\begin{verbatim}
port select --list gcc
\end{verbatim}
This returns quite the list:
\begin{verbatim}
Available versions for gcc:
        mp-gcc5
        mp-gcc7
        mp-gcc8
        mp-gcc9 (active)
        none
\end{verbatim}
When I type {\tt gcc} at the command prompt, it is the equivalent of typing
{\tt gcc-mp-9} (as indicated by ``active'' next to that version).  Changing
that is simple, but requires {\tt sudo} access:
\begin{verbatim}
sudo port select --set gcc mp-gcc8
\end{verbatim}
..and that's the gist of using \emph{select}.

Note that not all software is compatable with \emph{select} (in fact, most
isn't!).  MacPorts will complain if you try to do something that doesn't have
\emph{select} options.  Check to see if you have the corresponding
{\tt \*\_select} installed-- for example, if the above commands aren't working,
ensure that {\tt gcc\_select} is installed:
\begin{verbatim}
sudo port install gcc_select
\end{verbatim}

At any time, you can see the full status of all software that can be \emph{select}ed:
\begin{verbatim}
port select --summary
\end{verbatim}
This will tell you what software can be set via \emph{select}, what the options
are for each, and what is currently set:
\begin{verbatim}
Name                    Selected              Options
====                    ========              =======
clang                   none                  mp-clang-7.0 mp-clang-8.0 mp-clang-9.0 none
gcc                     mp-gcc9               mp-gcc5 mp-gcc7 mp-gcc8 mp-gcc9 none
ipython                 py38-ipython          py37-ipython py38-ipython none
ipython3                none                  py37-ipython py38-ipython none
mpi                     openmpi-gcc9-fortran  mpich-gcc5-fortran openmpi-gcc7-fortran openmpi-gcc9-fortran openmpi-mp-fortran none
pip3                    none                  pip37 none
python                  python38              python27 python36 python37 python38 none
python3                 python38              python36 python37 python38 none
spyder                  none                  spyder-37 none

\end{verbatim}

\subsection{Installing Python}
\label{sec:python}
Let's see how we install Python and all of its critical science
packages via MacPorts:
\begin{verbatim}
sudo port install python37
sudo port select --set python python37
sudo port install py37-ipython py37-scipy py37-numpy py37-matplotlib ipython_select
sudo port select --set ipython py37-ipython
\end{verbatim}

The {\tt select} commands make it so that we can call python and ipython
using their names, not their names and their versions (i.e., {\tt ipython-3.7}).

\subsection{Other Useful Packages}
Here's a quick list of some tools that are very useful.  They are listed by
their MacPorts name and suggested variants.
\begin{itemize}
  \item \textbf{\tt tkdiff}: a gui-based tool for comparing two text files.
  \item \textbf{\tt texlive}: Installs \LaTeX and some basic libraries.
    Most \LaTeX add-on libraries can be installed via MacPorts, too.
  \item \textbf{\tt TexShop}: a gui-based tool for editing \LaTeX files.  A
    clickable executable file will be placed in your Applications/MacPorts
    directory.
  \item \textbf{\tt cdf}: NASA's Common Data Format library.
  \item \textbf{\tt gcc49}: The Gnu C Compiler.  Includes gfortran, the leading
    open-sourced Fortran compiler.
  \item \textbf{\tt openmpi}: Open source Message Passing Interface library for
    compiling parallized software.
  \item \textbf{\tt ImageMagick +x11}: Powerful open-source image manipulation
    tools.
\end{itemize}

\section{Confirming Your Path}
At this point, you're nearly done, but it's a great idea to make sure that
your newly installed software is properly accessible. Remember that OS X has
its own non-standard versions of programs such as Python, so even though you're
using python, it may not be the right one, and only the right one will have
your extra installed goodies, such as Numpy.  Use {\tt which} to
ensure that you set your software path correctly.
\begin{verbatim}
which python
\end{verbatim}
If you see anything besides {\tt /opt/local/bin/python} here, be sure to
revisit setting up your search path in Sections \ref{sec:path} and
\ref{sec:python}.   Check for programs with a \emph{similar} name using
tab-complete (i.e., typing {\tt python} then pressing tab repeatedly to see
if there are other versions of python.)  Use {\tt which} on those programs.
Are they in the right place?  If so, use MacPorts' select syntax to set the
default version of python, as we did in Section \ref{sec:python}.

Note that the path returned by {\tt which} is \textbf{user dependent!}  This
includes {\tt root}!  A very common problem arises when users attempt to install
3rd party python packages.  The syntax to do this often looks like this,
\begin{verbatim}
python setup.py install
\end{verbatim}
...but this doesn't work due to insufficient permissions!  So, you try again:
\begin{verbatim}
sudo python setup.py install
\end{verbatim}
Note the use of {\tt sudo} to execute this command as root.  The command
executes successfully, but the python package cannot be imported when you run
python under your standard user account.  What happened?  The trick is to use
{\tt which} to learn which version of python was used by {\tt root} to install
the package:
\begin{verbatim}
sudo which python
\end{verbatim}
What is the output here?  Often, it's not the MacPorts install of Python, but
Apple's customized version ({\tt which} returns {\tt /usr/bin/python}).  There
are several ways to resolve this, including specfying the full path name of
the program (e.g., {\tt sudo /opt/local/bin/python setup.py install}) or editing
the config files for the different user.

\textbf{When performing an action using {\tt sudo} using MacPorts-installed
  software, always check your path as root!}

\section{Epilogue}
...and you're off!  Hopefully this primer has set you forward successfully.
Working from the command line is immensely powerful and becomes second nature
the more you use it.  It's not without its issues, though, so be ready for
more challenges.  In the end, it is an excellent way to get science done as
efficiently as possible.

For those who have never used a pure Linux machine, I urge you to find an
excuse to do so.  The steps outlined above are obnoxious enough to make
the novice user question the wisdom of turning to a command line approach.
However, setting up a real Linux system is so trivial and fast that it makes
you wonder why Apple doesn't include the command line tools, including package
managers, in a more native fashion.  On a system that fully embraces the
open-source approach, the whole process is elegant and fast.  Indeed, many
people who start with Macs wind up with a dedicated Linux machine on the side.

\end{document}
